% Options for packages loaded elsewhere
\PassOptionsToPackage{unicode}{hyperref}
\PassOptionsToPackage{hyphens}{url}
\PassOptionsToPackage{dvipsnames,svgnames,x11names}{xcolor}
%
\documentclass[
]{article}
\usepackage{amsmath,amssymb}
\usepackage{iftex}
\ifPDFTeX
  \usepackage[T1]{fontenc}
  \usepackage[utf8]{inputenc}
  \usepackage{textcomp} % provide euro and other symbols
\else % if luatex or xetex
  \usepackage{unicode-math} % this also loads fontspec
  \defaultfontfeatures{Scale=MatchLowercase}
  \defaultfontfeatures[\rmfamily]{Ligatures=TeX,Scale=1}
\fi
\usepackage{lmodern}
\ifPDFTeX\else
  % xetex/luatex font selection
\fi
% Use upquote if available, for straight quotes in verbatim environments
\IfFileExists{upquote.sty}{\usepackage{upquote}}{}
\IfFileExists{microtype.sty}{% use microtype if available
  \usepackage[]{microtype}
  \UseMicrotypeSet[protrusion]{basicmath} % disable protrusion for tt fonts
}{}
\makeatletter
\@ifundefined{KOMAClassName}{% if non-KOMA class
  \IfFileExists{parskip.sty}{%
    \usepackage{parskip}
  }{% else
    \setlength{\parindent}{0pt}
    \setlength{\parskip}{6pt plus 2pt minus 1pt}}
}{% if KOMA class
  \KOMAoptions{parskip=half}}
\makeatother
\usepackage{xcolor}
\usepackage[margin=1in]{geometry}
\usepackage{color}
\usepackage{fancyvrb}
\newcommand{\VerbBar}{|}
\newcommand{\VERB}{\Verb[commandchars=\\\{\}]}
\DefineVerbatimEnvironment{Highlighting}{Verbatim}{commandchars=\\\{\}}
% Add ',fontsize=\small' for more characters per line
\usepackage{framed}
\definecolor{shadecolor}{RGB}{248,248,248}
\newenvironment{Shaded}{\begin{snugshade}}{\end{snugshade}}
\newcommand{\AlertTok}[1]{\textcolor[rgb]{0.94,0.16,0.16}{#1}}
\newcommand{\AnnotationTok}[1]{\textcolor[rgb]{0.56,0.35,0.01}{\textbf{\textit{#1}}}}
\newcommand{\AttributeTok}[1]{\textcolor[rgb]{0.13,0.29,0.53}{#1}}
\newcommand{\BaseNTok}[1]{\textcolor[rgb]{0.00,0.00,0.81}{#1}}
\newcommand{\BuiltInTok}[1]{#1}
\newcommand{\CharTok}[1]{\textcolor[rgb]{0.31,0.60,0.02}{#1}}
\newcommand{\CommentTok}[1]{\textcolor[rgb]{0.56,0.35,0.01}{\textit{#1}}}
\newcommand{\CommentVarTok}[1]{\textcolor[rgb]{0.56,0.35,0.01}{\textbf{\textit{#1}}}}
\newcommand{\ConstantTok}[1]{\textcolor[rgb]{0.56,0.35,0.01}{#1}}
\newcommand{\ControlFlowTok}[1]{\textcolor[rgb]{0.13,0.29,0.53}{\textbf{#1}}}
\newcommand{\DataTypeTok}[1]{\textcolor[rgb]{0.13,0.29,0.53}{#1}}
\newcommand{\DecValTok}[1]{\textcolor[rgb]{0.00,0.00,0.81}{#1}}
\newcommand{\DocumentationTok}[1]{\textcolor[rgb]{0.56,0.35,0.01}{\textbf{\textit{#1}}}}
\newcommand{\ErrorTok}[1]{\textcolor[rgb]{0.64,0.00,0.00}{\textbf{#1}}}
\newcommand{\ExtensionTok}[1]{#1}
\newcommand{\FloatTok}[1]{\textcolor[rgb]{0.00,0.00,0.81}{#1}}
\newcommand{\FunctionTok}[1]{\textcolor[rgb]{0.13,0.29,0.53}{\textbf{#1}}}
\newcommand{\ImportTok}[1]{#1}
\newcommand{\InformationTok}[1]{\textcolor[rgb]{0.56,0.35,0.01}{\textbf{\textit{#1}}}}
\newcommand{\KeywordTok}[1]{\textcolor[rgb]{0.13,0.29,0.53}{\textbf{#1}}}
\newcommand{\NormalTok}[1]{#1}
\newcommand{\OperatorTok}[1]{\textcolor[rgb]{0.81,0.36,0.00}{\textbf{#1}}}
\newcommand{\OtherTok}[1]{\textcolor[rgb]{0.56,0.35,0.01}{#1}}
\newcommand{\PreprocessorTok}[1]{\textcolor[rgb]{0.56,0.35,0.01}{\textit{#1}}}
\newcommand{\RegionMarkerTok}[1]{#1}
\newcommand{\SpecialCharTok}[1]{\textcolor[rgb]{0.81,0.36,0.00}{\textbf{#1}}}
\newcommand{\SpecialStringTok}[1]{\textcolor[rgb]{0.31,0.60,0.02}{#1}}
\newcommand{\StringTok}[1]{\textcolor[rgb]{0.31,0.60,0.02}{#1}}
\newcommand{\VariableTok}[1]{\textcolor[rgb]{0.00,0.00,0.00}{#1}}
\newcommand{\VerbatimStringTok}[1]{\textcolor[rgb]{0.31,0.60,0.02}{#1}}
\newcommand{\WarningTok}[1]{\textcolor[rgb]{0.56,0.35,0.01}{\textbf{\textit{#1}}}}
\usepackage{graphicx}
\makeatletter
\def\maxwidth{\ifdim\Gin@nat@width>\linewidth\linewidth\else\Gin@nat@width\fi}
\def\maxheight{\ifdim\Gin@nat@height>\textheight\textheight\else\Gin@nat@height\fi}
\makeatother
% Scale images if necessary, so that they will not overflow the page
% margins by default, and it is still possible to overwrite the defaults
% using explicit options in \includegraphics[width, height, ...]{}
\setkeys{Gin}{width=\maxwidth,height=\maxheight,keepaspectratio}
% Set default figure placement to htbp
\makeatletter
\def\fps@figure{htbp}
\makeatother
\setlength{\emergencystretch}{3em} % prevent overfull lines
\providecommand{\tightlist}{%
  \setlength{\itemsep}{0pt}\setlength{\parskip}{0pt}}
\setcounter{secnumdepth}{-\maxdimen} % remove section numbering
\ifLuaTeX
  \usepackage{selnolig}  % disable illegal ligatures
\fi
\IfFileExists{bookmark.sty}{\usepackage{bookmark}}{\usepackage{hyperref}}
\IfFileExists{xurl.sty}{\usepackage{xurl}}{} % add URL line breaks if available
\urlstyle{same}
\hypersetup{
  pdftitle={NYDP Shooting Incident Data},
  colorlinks=true,
  linkcolor={Maroon},
  filecolor={Maroon},
  citecolor={Blue},
  urlcolor={blue},
  pdfcreator={LaTeX via pandoc}}

\title{NYDP Shooting Incident Data}
\author{}
\date{\vspace{-2.5em}2024-03-04}

\begin{document}
\maketitle

\hypertarget{dataset-description}{%
\subsection{Dataset Description}\label{dataset-description}}

This is a breakdown of every shooting incident that occurred in NYC
going back to 2006 through the end of the previous calendar year.

This data is manually extracted every quarter and reviewed by the Office
of Management Analysis and Planning before being posted on the NYPD
website. Each record represents a shooting incident in NYC and includes
information about the event, the location and time of occurrence. In
addition, information related to suspect and victim demographics is also
included. This data can be used by the public to explore the nature of
shooting/criminal activity. Please refer to the attached data footnotes
for additional information about this dataset.

Source:
\href{httpds://catalog.data.gov/dataset/nypd-shooting-incident-data-historic}{NYPD
Shooting Incident}

\hypertarget{step-1-install-andor-import-libraries}{%
\subsection{Step 1: Install And/Or Import
Libraries}\label{step-1-install-andor-import-libraries}}

(Optional): In case you didn't install any of these packages below,
please feel free to install it as it required, using these command
below.

\begin{Shaded}
\begin{Highlighting}[]
\CommentTok{\# install.packages("tidyverse")}
\CommentTok{\# install.packages("lubridate")}
\CommentTok{\# install.packages("ggplot2")}
\FunctionTok{library}\NormalTok{(tidyverse)}
\FunctionTok{library}\NormalTok{(lubridate)}
\FunctionTok{library}\NormalTok{(ggplot2)}
\end{Highlighting}
\end{Shaded}

\hypertarget{step-2-load-datatable}{%
\subsection{Step 2: Load datatable}\label{step-2-load-datatable}}

\begin{Shaded}
\begin{Highlighting}[]
\NormalTok{dataUrl }\OtherTok{\textless{}{-}} \StringTok{"https://data.cityofnewyork.us/api/views/833y{-}fsy8/rows.csv?accessType=DOWNLOAD"}

\NormalTok{shootingData }\OtherTok{=} \FunctionTok{read\_csv}\NormalTok{(dataUrl)}
\end{Highlighting}
\end{Shaded}

\begin{verbatim}
## Rows: 27312 Columns: 21
## -- Column specification --------------------------------------------------------
## Delimiter: ","
## chr  (12): OCCUR_DATE, BORO, LOC_OF_OCCUR_DESC, LOC_CLASSFCTN_DESC, LOCATION...
## dbl   (7): INCIDENT_KEY, PRECINCT, JURISDICTION_CODE, X_COORD_CD, Y_COORD_CD...
## lgl   (1): STATISTICAL_MURDER_FLAG
## time  (1): OCCUR_TIME
## 
## i Use `spec()` to retrieve the full column specification for this data.
## i Specify the column types or set `show_col_types = FALSE` to quiet this message.
\end{verbatim}

\begin{Shaded}
\begin{Highlighting}[]
\FunctionTok{glimpse}\NormalTok{(shootingData)}
\end{Highlighting}
\end{Shaded}

\begin{verbatim}
## Rows: 27,312
## Columns: 21
## $ INCIDENT_KEY            <dbl> 228798151, 137471050, 147998800, 146837977, 58~
## $ OCCUR_DATE              <chr> "05/27/2021", "06/27/2014", "11/21/2015", "10/~
## $ OCCUR_TIME              <time> 21:30:00, 17:40:00, 03:56:00, 18:30:00, 22:58~
## $ BORO                    <chr> "QUEENS", "BRONX", "QUEENS", "BRONX", "BRONX",~
## $ LOC_OF_OCCUR_DESC       <chr> NA, NA, NA, NA, NA, NA, NA, NA, NA, NA, NA, NA~
## $ PRECINCT                <dbl> 105, 40, 108, 44, 47, 81, 114, 81, 105, 101, 2~
## $ JURISDICTION_CODE       <dbl> 0, 0, 0, 0, 0, 0, 0, 0, 0, 0, 2, 0, 0, 0, 2, 2~
## $ LOC_CLASSFCTN_DESC      <chr> NA, NA, NA, NA, NA, NA, NA, NA, NA, NA, NA, NA~
## $ LOCATION_DESC           <chr> NA, NA, NA, NA, NA, NA, NA, NA, NA, "MULTI DWE~
## $ STATISTICAL_MURDER_FLAG <lgl> FALSE, FALSE, TRUE, FALSE, TRUE, TRUE, FALSE, ~
## $ PERP_AGE_GROUP          <chr> NA, NA, NA, NA, "25-44", NA, NA, NA, NA, "25-4~
## $ PERP_SEX                <chr> NA, NA, NA, NA, "M", NA, NA, NA, NA, "M", NA, ~
## $ PERP_RACE               <chr> NA, NA, NA, NA, "BLACK", NA, NA, NA, NA, "BLAC~
## $ VIC_AGE_GROUP           <chr> "18-24", "18-24", "25-44", "<18", "45-64", "25~
## $ VIC_SEX                 <chr> "M", "M", "M", "M", "M", "M", "M", "M", "M", "~
## $ VIC_RACE                <chr> "BLACK", "BLACK", "WHITE", "WHITE HISPANIC", "~
## $ X_COORD_CD              <dbl> 1058925.0, 1005028.0, 1007667.9, 1006537.4, 10~
## $ Y_COORD_CD              <dbl> 180924.0, 234516.0, 209836.5, 244511.1, 262189~
## $ Latitude                <dbl> 40.66296, 40.81035, 40.74261, 40.83778, 40.886~
## $ Longitude               <dbl> -73.73084, -73.92494, -73.91549, -73.91946, -7~
## $ Lon_Lat                 <chr> "POINT (-73.73083868899994 40.662964620000025)~
\end{verbatim}

\hypertarget{r-markdown}{%
\subsection{R Markdown}\label{r-markdown}}

This is an R Markdown document. Markdown is a simple formatting syntax
for authoring HTML, PDF, and MS Word documents. For more details on
using R Markdown see \url{http://rmarkdown.rstudio.com}.

When you click the \textbf{Knit} button a document will be generated
that includes both content as well as the output of any embedded R code
chunks within the document. You can embed an R code chunk like this:

\begin{Shaded}
\begin{Highlighting}[]
\FunctionTok{summary}\NormalTok{(cars)}
\end{Highlighting}
\end{Shaded}

\begin{verbatim}
##      speed           dist       
##  Min.   : 4.0   Min.   :  2.00  
##  1st Qu.:12.0   1st Qu.: 26.00  
##  Median :15.0   Median : 36.00  
##  Mean   :15.4   Mean   : 42.98  
##  3rd Qu.:19.0   3rd Qu.: 56.00  
##  Max.   :25.0   Max.   :120.00
\end{verbatim}

\hypertarget{including-plots}{%
\subsection{Including Plots}\label{including-plots}}

You can also embed plots, for example:

\includegraphics{NYPD-Shooting-Incident_files/figure-latex/pressure-1.pdf}

Note that the \texttt{echo\ =\ FALSE} parameter was added to the code
chunk to prevent printing of the R code that generated the plot.

\end{document}
